% Vorlage für verschiedenste Abschlussarbeiten angelehnt an das Corporate Design der Universität zu Lübeck
%
% geschrieben von Ronny Bergmann, 4. August 2013; erweitert von Jan Pascal Maas, 22.03.2015
%
% THE BEER-WARE LICENSE (Rev. 42):
% Ronny Bergmann <bergmann@math.uni-luebeck.de> wrote this file. As long as you
% retain this notice you can do whatever you want with this stuff. If we meet
% some day, and you think this stuff is worth it, you can buy me a beer or
% coffee in return.
% Jan Pascal Maas <maas@informatik.uni-luebeck.de> extended this file. As long as you
% retain this notice you can do whatever you want with this stuff. If we meet
% some day, and you think this stuff is worth it, you can buy me a beer or
% coffee in return.
%
% Dokumentklasse
\documentclass[
cleardoublepage=empty,	% leere Seite beim Kapitel komplett leer
fontsize=11pt,			% Schriftgröße: 11pt
a4paper,				% Format: DIN A4
toc=bibliography,		% Literatur ins Inhaltsverzeichnis
listof=leveldown,		% Einbinden von List of Figures/List of Tables ins Inhaltsverzeichnis
twoside,				% Zweiseitiges Layout
BCOR=13mm,				% Bundstegkorrektur
headinclude=true,		% Kopfzeile gehört zur Texthöhe
footinclude=false,		% ...Fußzeile nicht
parskip=half,			% zwischen zwei Absätzen eine halbe Zeilenhöhe Platz
DIV=10					% Typearea: Seiten in 10x10 teilen, um den Satzspiegel zu bestimmen
%,draft					% für schnellere Kompilierung das Kommentar entfernen, dann keine Grafiken 
]{scrbook}				% KOMA-Skript-Buchklasse
%
%
%
% Pakete (fold)
%
	%
	% Zeichensatz
	\usepackage[utf8]{inputenc} % Zeichensatz: UTF 8
	\usepackage[T1]{fontenc} % Schriftzeichensatz: europäisch
	\usepackage[german]{babel} % Neue Rechtschreibung und Silbentrennung
	%
	% Farben
	\usepackage[dvipsnames,table]{xcolor} % Farben
	%
	% Seitenlayout
	\usepackage{scrpage2}
	%
	% Grafiken
	\usepackage{graphicx}
	%
	% Titelanpassugnen
	\usepackage{titlesec,scalefnt}
	%
	% Inhaltsanpassungen
	\usepackage{titletoc}
	%
	% Mathematische Pakete
	\usepackage[fleqn]{amsmath}
	\usepackage{amssymb,amsthm}
	%
	% Bild- und Teilbildunterschriften (hypcap setzt den Anker vernünftig
	\usepackage[hypcap=true]{caption}
	\usepackage[hypcap=true]{subcaption}
	%
	% Listen
	\usepackage{enumitem}
	\usepackage{tikz}
	%
	% Tabellen
	\usepackage{booktabs,colortbl,longtable}
	%
	% Quelltext-Einbinden
	\usepackage[final]{listings}
	%
    % Für zusätzliche Titelseite mit anderen Maßen
    \usepackage{geometry}
	% Hyperref - load after pagestyle due to captionsredefinition there
	\usepackage{url}
	\usepackage{hyperref}
	\usepackage[all]{hypcap}	% Link to start of figures, not captions
	%
	% Literatur
	\usepackage[numbers,sort]{natbib}
% (end)
%
%
%
% Einstellungen & globale Variablen (fold)
    % Name, Titel und andere Einstellungen
    \newcommand{\UzLAutor}{Dein Name} 	% Name
    \newcommand{\UzLAutorKurz}{D. Name} % Name short
    \newcommand{\UzLStudiengang}{Studiengang} % Degree course
    \newcommand{\UzLTitel}{Dein Thema} %Title
    \newcommand{\UzLTitelFormatiert}{%
        \LARGE\bfseries%
        Dein%
        % Zeilenumbruch (Line break)
        \\[.5\baselineskip]%
        Thema
    } %Title with adapted line breaks
    \newcommand{\UzLTitelEN}{Your Theme} %Title in English
    \newcommand{\UzLTitelFormatiertEN}{%
        \LARGE%
        Your%
        % Zeilenumbruch (Line break)
        \\[.5\baselineskip]%
        Theme
    } %Title in English with adapted line breaks
    \newcommand{\UzLArbeitstyp}{Art der Arbeit}
    \newcommand{\UzLDatum}{Datum ohne Tag} %Date (Month&Year)
    \newcommand{\UzLgenauesDatum}{Datum mit Tag} %Date complete
    % Üblicherweise Lübeck
    \newcommand{\UzLOrt}{Lübeck} %Place you wrote the thesis
    % Prüfer und Unterstützer
    \newcommand{\UzLPruefer}{Dein Prüfer}
    \newcommand{\UzLUnterstuetzera}{Dein erster Unterstützer}
    \newcommand{\UzLUnterstuetzerb}{Dein zweiter Unterstützer}
    % Insitut und Leitung
    \newcommand{\UzLInstitut}{Dein Institut}
    \newcommand{\UzLInstitutsLeitung}{Dein Institutsleiter}
    %
	% Farbe
	\definecolor{UzLcolor}{cmyk}{1,.0,.20,.78} % Universitätsfarbe in CMYK
%	\definecolor{UzLcolor}{cmyk}{0,0,0,.9} % In Graustufen diese Alternative wählen
	%
% (end)
%
%
%
% Seitenlayout (fold)
%
	%
	% Seitenlayout setzen
	\pagestyle{scrheadings}
	\clearscrheadings
	%innen: Kapitel Section
	\rehead{\leftmark}\lohead{\rightmark}
	%außen Seitenzahlen
	\rofoot{\pagemark}\lefoot{\pagemark}
	%
	%
	% Schriftart für Überschriften auf serifen, nur bis subsubsection nummerieren
	\setkomafont{sectioning}{\rmfamily\color{UzLcolor}\bfseries}
	\setkomafont{title}{\color{UzLcolor}}
	\setcounter{secnumdepth}{1} % Part/Chapter/Section mit Nummern 
	%
	% Listen-Einstellungen
	\setlist{itemsep=.5\baselineskip,leftmargin=1.5em}
	\setlist[enumerate,1]{label=\alph*)}
	%
	%
	% Vermeidung von Hurenkindern und Schusterjungen
	\clubpenalty=10000
	\widowpenalty=10000
	\displaywidowpenalty=10000
	%
	% Eigene Theorema-Umgebungen
%	\newtheoremstyle{normalstyle}% name
%		{\baselineskip}%	Space above
%		{}%	Space below
%		{\normalfont}%	Body font
%		{}%	Indent amount
%		{\normalfont\bfseries}% Theorem head font
%		{.\newline}%	Punctuation after theorem head
%		{.5\baselineskip}%	Space after theorem head
%		{}%	Theorem head spec (can be left empty, meaning ‘normal’)
%	\theoremstyle{normalstyle}
%	\newtheorem{thm}{Theorem}[chapter]
%	\newtheorem{lem}[thm]{Lemma}
%	\newtheorem{kor}[thm]{Korollar}
%	\newtheorem{definition}[thm]{Definition}
%	\newtheorem{rem}[thm]{Bemerkung}
%	\newtheorem{ex}[thm]{Beispiel}
	%
	% Abbildungsformatierung
	\DeclareCaptionLabelSeparator{periodspace}{.\ }
	\captionsetup{format=hang,labelsep=periodspace,indention=-2cm,labelfont=bf,width=.9\textwidth,skip=.5\baselineskip}
	\captionsetup[table]{position=above}
	\captionsetup[sub]{labelfont=bf,labelsep=period,labelformat=mysublabelfmt,subrefformat=simple}
	%
	% Optik der Quelltexte mit eigenem Stil
	\renewcommand{\lstlistlistingname}{Algorithmenverzeichnis}
	\renewcommand{\lstlistingname}{Algorithmus}
	\lstset{language=Mathematica}
	\lstset{basicstyle={\sffamily\footnotesize},numbers=left,numberstyle=\tiny\color{UzLcolor},numbersep=5pt,
	    breaklines=true,
	    captionpos={t},
	    frame={lines},rulecolor=\color{black},framerule=0.5pt,
		mathescape, columns=flexible,
		tabsize=2
	}
	\lstdefinestyle{mystyle}
	{
		keywordstyle=\bfseries,
		numbers=left,
		numberstyle=\tiny\color{UzLcolor},
		numbersep=5pt,
	    breaklines=true,
		frame={lines},
		rulecolor=\color{UzLcolor},
		framerule=0.5pt,
		backgroundcolor=\color{UzLcolor!5},
		aboveskip=1em,
		belowskip=1em,
		showstringspaces=false,
		tabsize=3,
		framesep=0.75em
	}	
	%
	% Chapter Style
	\titleformat{\chapter}[display]%
	    {\relax%
			\raggedleft%
			\huge\bfseries
			\color{UzLcolor}%
		}%
	    {\Huge\raggedleft
		\raggedleft{\textcolor{UzLcolor!25}{\scalefont{8}\thechapter}}}%
	    {-.5\baselineskip} %
	    {}
	\titlespacing{\chapter}{0pt}{\baselineskip}{\baselineskip}
	%
	% Design of the table of contents
	\titlecontents{chapter}[.5em]{\vspace{1\baselineskip}}{\contentslabel{1em}\large}{\large%\hspace*{-3.2em}
		}{\titlerule*[0.5pc]{}{\large\contentspage}}[\vspace{.1\baselineskip}]
	\titlecontents{section}[2.5em]{}{\contentslabel{2em}}{%\hspace*{-3.2em}
		}{\titlerule*[0.5pc]{.}\contentspage}
	\titlecontents{subsection}[3.5em]{}{\contentslabel{1.5em}}{%\hspace*{-3.2em}
		}{\titlerule*[0.5pc]{.}\contentspage}
	%
	\makeatletter% --> Anpassung der Nummernbreite
	\renewcommand*{\@pnumwidth}{1.7em}
	\makeatother% --> \makeatletter
% (end)
%
%
%
% PDF-Meta-Informationen (fold)
	\hypersetup{
		pdfauthor={\UzLAutor},
		pdftitle={\UzLTitel},
		pdfsubject={Bachelorarbeit},
%    	pdfsubject={Masterarbeit},
		pdfproducer={pdfLaTeX},
		pdfcreator={YourEditor},
		unicode=true,
		pdfencoding=auto,
		breaklinks=true,
		plainpages=false,
		pdfstartview=FitH,
		pdfview=FitH,
		pdfpagemode=UseOutlines,
		bookmarksnumbered=true,
		bookmarksopen=true,
		bookmarksopenlevel=1,
		pdfdisplaydoctitle=true,
		pdfduplex=true,
		pdflang=de,
		colorlinks=true,
		linkcolor=UzLcolor,
		citecolor=black,
		urlcolor=black
	}
% (end)
%
%
%
% Dokumentanfang (fold)
\begin{document}
	%
	%
	% Abschnitt vor dem eigentlichen Text
	\frontmatter
    %---------------------------------------------------------------------------
    % Frontpage
    %---------------------------------------------------------------------------
    %---------------------------------------------------------------------------
% Frontpage
%---------------------------------------------------------------------------

% Die Richtline zum Aufbau des Deckblatts von Bachelor- und Masterarbeiten
% findet sich hier:
% @see: http://www.uni-luebeck.de/fileadmin/uzl_ssc/PDF-Dateien/Richtlinie-Deckblatt-MINT-Abschlussarbeit-2012-10-18.pdf

\newcommand{\titlepageskip}{\vskip 20pt}

% @see: http://tex.stackexchange.com/questions/31705/different-margins-for-title-page
\newgeometry{top=1in,bottom=1in,right=1in,left=1.2in}
\begin{titlepage}
    
    \title{dein Titel}
    \author{Jan Pascal Maas}
    
    {\Large
        % 1. Offizielles Logo des Instituts, an dem die Arbeit angesiedelt ist. (Das offizielle Logo
        % enth�lt das Siegel der Universit�t zusammen mit dem Text "Universit�t zu L�beck"
        % und darunter den Namen des Instituts.) Dieses Logo ist bei den Instituten zu
        % bekommen. Das Logo muss oben links platziert werden.
        \includegraphics[width=140mm, height=50px, draft]{images/institutslogo.pdf}
        \vskip 44pt
        
        % 2. Optional: Noch einmal Name des Instituts und Angabe der Direktorin oder des
        % Direktors des Instituts.
%        \UzLInstitut ~unter der Leitung von \UzLInstitutsLeitung
        
        % 3. Titel der Arbeit in deutscher Sprache und ebenfalls in englischer Sprache. Dabei soll
        % die Sprache, in der die Arbeit verfasst wurde, als erste angef�hrt werden; die andere
        % Sprache kann weniger prominent dargestellt werden.
        % Auch bei englischsprachigen Studieng�ngen sollen die Titelbl�tter auf Deutsch sein.
        {\UzLTitelFormatiert\par}
        {\UzLTitelFormatiertEN\par}
        
        \titlepageskip
        % 4. Der Text "Bachelorarbeit" oder "Masterarbeit" (nicht "Bachelor-Arbeit" oder "Master-Arbeit").
        {\bfseries \UzLArbeitstyp}
        
        \titlepageskip
        %5. Der Text "im Rahmen des Studiengangs"
        im Rahmen des Studiengangs\\
        %6. Der ausgeschriebene Name des Studiengangs (also beispielsweise "Informatik"
        %oder "Molecular Life Science", hingegen nicht "Bioinformatik" oder "MLS")
        {\bfseries \UzLStudiengang}\\
        %7. Der Text "der Universit�t zu L�beck"
        der Universit"at zu L"ubeck
        
        \titlepageskip
        %8. Der Text "Vorgelegt von" und der Name der Studentin oder des Studenten
        vorgelegt von\\
        {\bfseries \UzLAutor}
        
        \titlepageskip
        %9. Der Text "Ausgegeben und betreut von"
        ausgegeben und betreut von\\
        %10. Der Name der ersten Pr�ferin oder des ersten Pr�fers. Dies ist immer gleichzeitig
        %die Betreuerin oder der Betreuer im Sinne der Pr�fungsordnung.
        {\bfseries \UzLPruefer}
        
        % Diesen Teil entfernen, wenn die Arbeit KEINEN Unterst�tzer hatte
       	\titlepageskip
       	{
       		%11. Optional der Text "Mit Unterst�tzung von" und der Name von weiteren Personen,
       		%die die Betreuung besonders unterst�tzt haben. Beispielsweise k�nnen dies
       		%wissenschaftliche Mitarbeiter sein oder Mitarbeiter von Firmen, wenn die Arbeit
       		%extern geschrieben wurde.
       		mit Unterst"utzung von
       		
    		{\bfseries \UzLUnterstuetzera}\\
    		{\bfseries \UzLUnterstuetzerb}\\
       	}
        
        \vfill 
        %13. Der Text "L�beck, den" und das Abgabedatum.
        {
            \UzLOrt, den \UzLDatum
        }
        
        % Diesen Teil entfernen, wenn "Im Focus das Leben" nicht drauf stehen soll
        %14. Optional der Text "Im Focus das Leben".
        {
            \titlepageskip
            Im Focus das Leben
        }
    }
\end{titlepage}
\restoregeometry

\cleardoublepage

% Erklaerung
\newpage
\vspace*{7cm}
\centerline{\bfseries Erkl"arung}

\vspace*{1cm}
Ich versichere an Eides statt, die vorliegende Arbeit selbstst"andig und nur unter Benutzung
der angegebenen Hilfsmittel angefertigt zu haben.

\vspace*{3cm}
\UzLOrt, den \UzLDatum

\pagestyle{scrheadings}

\cleardoublepage

% Kurzfassung und Abstract

\section*{Kurzfassung}

\input{0-kurzfassung}

%
\vskip 3cm
%

\section*{Abstract}

\input{0-abstract}

\cleardoublepage

%% Aufgabenstellung (Optional)
%
%\section*{Aufgabenstellung}
%
%Die Aufgabenstellung der Master-/Diplom-/Bachelor-/Studienarbeit

    
    %---------------------------------------------------------------------------
    % Inhaltsverzeichnis
    %---------------------------------------------------------------------------
    \tableofcontents
    \cleardoublepage
    %---------------------------------------------------------------------------
    % Der eigentliche Inhalt
    %---------------------------------------------------------------------------
	\mainmatter
	\chapter{Einleitung}
	Dies ist die Einleitung. Und ein Zitat~\citep{TestEntry}
	\chapter{Grundlagen}
	Grundlagen
	% (end)
	%
	% Bibliothek
	\bibliographystyle{abbrv}
	\bibliography{literatur}
	%
	% weitere Verzeichnisse
	\addchap{Verzeichnisse}
	\begingroup
		\let\chapter=\section
		%
		\listoffigures %Abbildungsverzeichnis
%		\lstlistoflistings %Quelltextverzeichnis
		\listoftables %Tabellenverzechnis
		% Add further lists here
	\endgroup
	\appendix
\end{document}
% (end)
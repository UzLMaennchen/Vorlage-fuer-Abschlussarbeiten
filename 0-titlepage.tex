%---------------------------------------------------------------------------
% Frontpage
%---------------------------------------------------------------------------

% Die Richtline zum Aufbau des Deckblatts von Bachelor- und Masterarbeiten
% findet sich hier:
% @see: http://www.uni-luebeck.de/fileadmin/uzl_ssc/PDF-Dateien/Richtlinie-Deckblatt-MINT-Abschlussarbeit-2012-10-18.pdf

\newcommand{\titlepageskip}{\vskip 20pt}

% @see: http://tex.stackexchange.com/questions/31705/different-margins-for-title-page
\newgeometry{top=1in,bottom=1in,right=1in,left=1.2in}
\begin{titlepage}
    
    \title{dein Titel}
    \author{Jan Pascal Maas}
    
    {\Large
        % 1. Offizielles Logo des Instituts, an dem die Arbeit angesiedelt ist. (Das offizielle Logo
        % enth�lt das Siegel der Universit�t zusammen mit dem Text "Universit�t zu L�beck"
        % und darunter den Namen des Instituts.) Dieses Logo ist bei den Instituten zu
        % bekommen. Das Logo muss oben links platziert werden.
        \includegraphics[width=140mm, height=50px, draft]{images/institutslogo.pdf}
        \vskip 44pt
        
        % 2. Optional: Noch einmal Name des Instituts und Angabe der Direktorin oder des
        % Direktors des Instituts.
%        \UzLInstitut ~unter der Leitung von \UzLInstitutsLeitung
        
        % 3. Titel der Arbeit in deutscher Sprache und ebenfalls in englischer Sprache. Dabei soll
        % die Sprache, in der die Arbeit verfasst wurde, als erste angef�hrt werden; die andere
        % Sprache kann weniger prominent dargestellt werden.
        % Auch bei englischsprachigen Studieng�ngen sollen die Titelbl�tter auf Deutsch sein.
        {\UzLTitelFormatiert\par}
        {\UzLTitelFormatiertEN\par}
        
        \titlepageskip
        % 4. Der Text "Bachelorarbeit" oder "Masterarbeit" (nicht "Bachelor-Arbeit" oder "Master-Arbeit").
        {\bfseries \UzLArbeitstyp}
        
        \titlepageskip
        %5. Der Text "im Rahmen des Studiengangs"
        im Rahmen des Studiengangs\\
        %6. Der ausgeschriebene Name des Studiengangs (also beispielsweise "Informatik"
        %oder "Molecular Life Science", hingegen nicht "Bioinformatik" oder "MLS")
        {\bfseries \UzLStudiengang}\\
        %7. Der Text "der Universit�t zu L�beck"
        der Universit"at zu L"ubeck
        
        \titlepageskip
        %8. Der Text "Vorgelegt von" und der Name der Studentin oder des Studenten
        vorgelegt von\\
        {\bfseries \UzLAutor}
        
        \titlepageskip
        %9. Der Text "Ausgegeben und betreut von"
        ausgegeben und betreut von\\
        %10. Der Name der ersten Pr�ferin oder des ersten Pr�fers. Dies ist immer gleichzeitig
        %die Betreuerin oder der Betreuer im Sinne der Pr�fungsordnung.
        {\bfseries \UzLPruefer}
        
        % Diesen Teil entfernen, wenn die Arbeit KEINEN Unterst�tzer hatte
       	\titlepageskip
       	{
       		%11. Optional der Text "Mit Unterst�tzung von" und der Name von weiteren Personen,
       		%die die Betreuung besonders unterst�tzt haben. Beispielsweise k�nnen dies
       		%wissenschaftliche Mitarbeiter sein oder Mitarbeiter von Firmen, wenn die Arbeit
       		%extern geschrieben wurde.
       		mit Unterst"utzung von
       		
    		{\bfseries \UzLUnterstuetzera}\\
    		{\bfseries \UzLUnterstuetzerb}\\
       	}
        
        \vfill 
        %13. Der Text "L�beck, den" und das Abgabedatum.
        {
            \UzLOrt, den \UzLDatum
        }
        
        % Diesen Teil entfernen, wenn "Im Focus das Leben" nicht drauf stehen soll
        %14. Optional der Text "Im Focus das Leben".
        {
            \titlepageskip
            Im Focus das Leben
        }
    }
\end{titlepage}
\restoregeometry

\cleardoublepage

% Erklaerung
\newpage
\vspace*{7cm}
\centerline{\bfseries Erkl"arung}

\vspace*{1cm}
Ich versichere an Eides statt, die vorliegende Arbeit selbstst"andig und nur unter Benutzung
der angegebenen Hilfsmittel angefertigt zu haben.

\vspace*{3cm}
\UzLOrt, den \UzLDatum

\pagestyle{scrheadings}

\cleardoublepage

% Kurzfassung und Abstract

\section*{Kurzfassung}

\input{0-kurzfassung}

%
\vskip 3cm
%

\section*{Abstract}

\input{0-abstract}

\cleardoublepage

%% Aufgabenstellung (Optional)
%
%\section*{Aufgabenstellung}
%
%Die Aufgabenstellung der Master-/Diplom-/Bachelor-/Studienarbeit
